\section{Introduction}
\label{sec:Introduction}
Heavy quarkonium has long served as a probe for quantum chromodynamics (QCD). The production of heavy quarkonium can be separated into two steps: First, quark and anti-quark pairs $q\bar{q}$ with certain color and spin are produced in hard collisions, which can be calculated perturbatively Then, the produced $q\bar{q}$ hadronize into quarkonium at a momentum scale smaller than the mass of quarkonium. The second step is non-perturbative. Different treatments are proposed to solve the non-perturbative process, leading to differnt models like Color-Singlet Model (CSM), Color-Octet Model~\cite{Bodwin:1992qr}, Color-Evaporation Model (CEM)~\cite{Halzen:1977rs}, non-relavatistic QCD (NRQCD)~\cite{Bodwin:1994jh} and so on. 

Quarkonium production is influenced by various mechanisms. In the very beginning stage of collisions, the color and momentum distributions of partons, which governed by the nuclear parton distribution function (nPDF)~\cite{AtashbarTehrani:2017mzi} will affect the quarkonium production. Geometric of the colliding nucleus also take part in the modification due to the resulting energy loss by multi-particle scattering~\cite{Arleo:2014oha}. These effects function ahead of the quarkonium production in the initial stage is called initial-state effects. After heavy quarkonium is produced, it may also go through break-up and recombination, resulting in enhancement or suppression due to different effects. For example, quarkonium may break up via the multi-scattering in hot and dense quark-gluon plasma (QGP)~\cite{Wu:2010ze}. The Debye color screening~\cite{Quigg:1979vr} from the extremely dense color charge will also lead to the dissociation of heavy quarkonium. The dissociation due to the interaction with colliding nucleus is called nuclear absorption~\cite{Gavin:1990gm}. Luckily, in the rest frame of target nucleus at high LHC energy, the formation time of
quarkonium will boost to be larger than the radii of nucleus. In this case, the nuclear absorption becomes negligible. The produced quarkonium will also interact with the comoving particles in the similar rapidity ranges, named comover effect~\cite{Ferreiro:2012rq}. These effects, function after the quarkonium is produced, are final-state effects. 

By studying the $\psitwos-to-\jpsi$ ratio, the initial-state effects cancel. Due to the larger radius of \psitwos, it will dissociate more easily in QGP and in the interaction with comoving particles. Therefore, both QGP and comover effect are multiplicity-dependent. Research on the multiplicity dependence of $\psitwos-to-\jpsi$ ratio help to throw lights on the effects by the potentially existing QGP and verify the effectiveness of the comover model. The co-mover model has been successful at describing suppression of charmonia, bottomonia, and $X(3872)$ hadrons in various collision systems ~\cite{Gavin:1996yd,Ferreiro:2018wbd,Esposito:2020ywk,Braaten:2020iqw}. 


Another mechanism that affects the quarkonium production is the co-mover effect~\cite{Ferreiro:2012rq}, in which co-moving particles can modify quarkonium properties, resulting in the suppression of these states. Excited states of quarkonium are loosely bound with larger radii, and are more likely to dissociate in the interaction with co-moving particles, which manifests as relative suppression compared to ground states. The suppression by interaction with co-moving particles is correlated to the charged particle multiplicity since the latter influences the number of co-moving particles. The co-mover model has been successful at describing suppression of charmonia, bottomonia, and $X(3872)$ hadrons in various collision systems ~\cite{Gavin:1996yd,Ferreiro:2018wbd,Esposito:2020ywk,Braaten:2020iqw}. 

Multiplicity dependence of $\psitwos-to-\jpsi$ production ratio has also been studied by different experiments. First measurement is given by NA50 experiment PbPb collisions~\cite{NA50:2006yzz}. A decreasing trend for $\psitwos-to-\jpsi$ ratio with transverse energy, which is roughly proportional to the multiplicity, is found. However, ALICE did not find multiplicity dependence~\cite{Hushnud:2023kwf} in PbPb collisions. Then ALICE also measure the $\psitwos-to-\jpsi$ production ratio in $pp$ and $p$Pb collisions~\cite{ALICE:2022gpu}, which didn't find a multiplicity dependence, either. The different behaviours may arise from the different multiplicity schemes from ALICE and NA50 experiments. Meanwhile, the  $\psitwos-to-\jpsi$ ratio will also behave differently in different collision systems. 

This study uses $p$Pb collision data collected in 2016 by the LHCb detector at a center-of-mass energy of 8.16 TeV, corresponding to a luminosity of 13.6$\pm$0.3 pb$^{-1}$ (20.8$\pm$0.5 pb$^{-1}$ for Pb$p$), to measure and compare the production cross-sections of prompt and non-prompt (from $b-$hadron decay) \psitwos and \jpsi mesons.  The subsequent determination of cross-section ratios across different multiplicity regions help to understand the relation between charmonium production, QGP and the comover effect. 

To distinguish between the influence of the comover effect on the production and that of QGP, several types of multiplicity variables are needed. In this analysis, the \psitwos-to-\jpsi cross-section ratio is measured as a function of local multiplicity, which is the multiplicity close to PV, as well as forward (backward) multiplicity, which is measured in same (opposite) rapidity range over which the charmonia production is measured. The usage of backward multiplicity allows the co-mover effect to be removed so that the existence of QGP can be tested.