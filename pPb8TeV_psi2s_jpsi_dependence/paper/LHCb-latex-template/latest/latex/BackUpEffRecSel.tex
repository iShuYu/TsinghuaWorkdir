The tracking efficiency is computed using the simulation samples, and this efficiency is corrected using the reconstruction efficiency per-track estimated in data. The long method with tag-probe strategy for tracking efficiency calibration is implemented. In this method, a probe track is reconstructed only with hits in the TT and MUON stations. This probe muon track is combined with a standard long muon track to form a \jpsi candidate, and these candidates build a “pre-matched” sample. This standard long track has the same reconstruction algorithm as the signal tracks in our analysis, so they have the same tracking efficiency when they are in the same phase space. 
An additional standard long track, identified as a muon and with the same charge as the probe track, is combined with the \jpsi in the ”pre-matched” sample to form a good vertex. If this third track shares with the probe track more than $40\%$ of the hits in both the TT and MUON stations, the probe track is referred to as “matched”. The \jpsi candidates in the ”pre-match” sample that have the probe track matched, form the ”matched” sample. 
The tracking efficiency is computed as the matching efficiency, which is the fraction of probe tracks that match standard long tracks. The number of signal probe tracks and those matched to long tracks are estimated from the number of \jpsi signal candidates, measured by fitting to the $\mu^+\mu^-$ invariant mass distribution in the ”pre-matched” sample and the ”matched” sample, respectively. The reconstruction of the probe tracks, of the \jpsi candidates and the implementation of the matching are done at the trigger (HLT) level, available in both pp and proton-lead data. The relevant trigger lines are called $\textbf{Hlt2TrackEffDiMuonMuonTT(Minus|Plus)^*}$ (which selects the “pre-matched” samples), where Plus (Minus) means that the $\mu^+(\mu^-)$ is the probe track. These calibration events are processed via the TurboCalib stream. Due to higher multiplicity in proton-lead data, especially in the backward configuration, additional offline selections are applied to the \jpsi candidates out of TurboCalib stream. The tag track of the \jpsi candidates is required to have a good muon-pion separation with $PID(\mu)$ > 3, and the probe track is required to have a better fit quality with $Prob(\chi_{trk}^2) > 0.2$. The effect of these extra selections is studied with the proton-lead forward data sample ($p$Pb), where a better signal purity is obtained. 
Simulated \jpsi samples are produced with the same trigger processing as data for the calibration. The reconstruction software is identical to those used to reconstruct the simulated signals used in the analysis. 
The signal extraction fits to the mass distributions are implemented in bins of $\eta$ and $p$ of the probe tracks, allowing to determine the track reconstruction efficiency in the same bins. The bin boundaries are 1.9, 3.2 and 5.0 for $\eta$ and 6, 10, 20, 40, 100 and 500 $\gevc$ for $p$. No binning in detector occupancy is implemented due to limited statistics. However, since for both data and simulation, the occupancy distribution in the analysis samples and the calibration samples are consistent with each other,  the binning in detector occupancy is not necessary. 
The $\mu^+$ and $\mu^-$ probe tracks are fit separately. Thus for each kinematic bins, there are eight fits: $\mu^+$ or $\mu^-$ as the probe track; in the ”pre-matched” or ”matched” sample; for $p$Pb or Pb$p$. For the fits, the same signal shape is used for bins of the same $(p, \eta)$ interval: a Gaussian function plus a Crystal Ball function. The background is described by an exponential function. 
The procedure performed on data is applied identically to the $p$Pb and Pb$p$ simulation calibration samples. The tracking efficiencies for $\mu^+$ and $\mu^-$ are averaged, as for the cross-section measurements charge conjugated states are added together.
Finally, we calculate the ratio of single track reconstructions efficiencies between data and simulation. The results are given in Figure~{TrackTable}. The corresponding uncertainty for each value is also shown. The ratio reconstruction and selection efficiency is then computed from the full simulation, and correcting the efficiency using the per-track efficiency ratios detailed above.