\section{Detector and data samples}
\label{Data and Monte Carlo samples}
\subsection{LHCb Detector}
The LHCb detector~\cite{LHCb-DP-2008-001,LHCb-DP-2014-002} is a single-arm forward spectrometer covering the pseudorapidity range $2 < \eta < 5$, designed for the study of particles containing $b$ or $c$ quarks. The detector includes a high-precision tracking system consisting of a silicon-strip vertex detector surrounding the $pp$ interaction region~\cite{LHCb-DP-2014-001}, a large-area silicon-strip detector located upstream of a dipole magnet with a bending power of about 4 Tm, and three stations of silicon-strip detectors and straw drift tubes~\cite{LHCb-DP-2013-003} placed downstream of the magnet. The tracking system provides a measurement of momentum, $p$, of charged particles with a relative uncertainty that varies from $0.5\%$ at low momentum to $1.0\%$ at 200 \gevc. The minimum distance of a track to a primary vertex, the impact parameter, is measured with a resolution of $(15 + 29/\pt) \mu m$, where \pt is expressed in \gevc. Different types of charged hadrons are distinguished using information from two ring-imaging Cherenkov detectors~\cite{LHCb-DP-2012-003}. Photons, electrons and hadrons are identified by a calorimeter system consisting of scintillating-pad (SPD) and preshower detectors, an electromagnetic calorimeter and a hadronic calorimeter. Muons are identified by a system composed of alternating layers of iron and multiwire proportional chambers~\cite{LHCb-DP-2012-002}.

\subsection{Data samples}
The study conducted here utilizes $p$Pb collision data from 2016, acquired by the LHCb detector at a center-of-mass energy of 8.16 TeV. The corresponding luminosity is 13.6 $\pm$ 0.3 nb$^{-1}$ for the $p$Pb configuration and 20.8 $\pm$ 0.5 nb$^{-1}$ for the Pb$p$ configuration.

To investigate efficiencies, Monte Carlo simulation samples for \jpsi and \psitwos are generated. Within this simulation, $p$Pb collisions are produced through Pythia~\cite{Ball:2006wn}, employing a specific LHCb configuration~\cite{LHCb:2011dpk}. For the description of hadronic particle decays, EvtGen~\cite{Lange:2001uf} is utilized, incorporating final-state radiation via Photos~\cite{Golonka:2005pn}. The interaction between the simulated particles and the detector, as well as the detector's response, are modeled using the Geant4 toolkit~\cite{GEANT4:2002zbu}, as outlined in Ref.~\cite{Clemencic:2011zza}.

Signal samples encompassing $\jpsi \rightarrow \mu^+\mu^-$ and $\psitwos \rightarrow \mu^+\mu^-$ are generated through an embedding technique: minimum bias events are created using the Pythia (version 8) generator, simulating proton beams colliding with momenta equivalent to the momenta per nucleon of the heavy ion beams or targets. Subsequently, the \jpsi (\psitwos) mesons are extracted from these minimum bias events, discarding all other particles within the events.Their decays are forced to the signal decay modes using the EvtGen package The resultant decay chain is then added to a single minimum bias Epos event generated with beam parameters identical to those seen in data.


