
Your introduction provides a good overview of the topic of charmonia production in collisions and its relevance for studying the Quark-Gluon Plasma (QGP). However, there are a few ways you can improve the clarity and coherence of your paragraphs. I'll provide suggestions for each section:

Introduction to Quark-Gluon Plasma and Heavy-Ion Collisions:
Quark-Gluon Plasma (QGP)~\cite{Satz_2011} represents an extraordinary form of matter characterized by extremely high temperatures and densities. Within the QGP state, quarks and gluons, which are fundamental constituents of matter, become deconfined and exhibit asymptotic freedom. This remarkable phase is believed to have existed shortly after the Big Bang, during the early stages of the universe. To explore and study the properties of QGP, researchers turn to heavy-ion collisions, a setting that generates an environment of exceptional heat and density due to ultra-relativistic collision energies. In this context, the production of quarkonia, particularly charmonia, in hadronic collisions emerges as a valuable probe for investigating QGP, given the remarkably short production time compared to the QGP formation timescale.

Effects on Quarkonium Production:
The production of quarkonia is intricately influenced by the entire evolution of heavy-ion collisions. Consequently, the initial-state conditions, including nuclear parton distribution functions (nPDFs) for lead (Pb) nuclei and parton distribution functions for protons~\cite{AtashbarTehrani:2017mzi}, as well as the energy loss due to multi-particle scattering~\cite{Arleo:2014oha}, impose crucial effects on quarkonium production. These effects can lead to enhancements or suppressions in quarkonium yields. While considering the ratio of excited-to-ground state quarkonia production, some of these initial-state effects tend to cancel out. However, observations indicate that excited states of charmonia (\psitwos) and bottomonia (\TwoS, \ThreeS) are preferentially suppressed compared to their ground states (\jpsi and $\Upsilon$(1S))\cite{LHCb:2018psc,ALICE:2020vjy, LHCb:2016vqr}, indicating the involvement of additional, final-state effects such as interactions with co-moving particles\cite{Ferreiro:2012rq}.

Experimental Observations and Sensitivity:
Recent measurements of \jpsi and \psitwos production, as functions of collision centrality in $p$Pb collisions at $\sqrt{s_{NN}}=8.16$ TeV~\cite{ALICE:2020tsj}, underscore that \psitwos exhibits heightened sensitivity to final-state interactions. Notably, the nuclear modification of \psitwos is notably lower than that of \jpsi, particularly in all but the lowest and highest multiplicity classes. Similarly, the CMS collaboration has explored the event activity dependence of $\Upsilon$(nS) production in $p$Pb collisions at $\sqrt{s_{NN}}=5.02$ TeV and $pp$ collisions at $\sqrt{s}=2.76$ TeV~\cite{CMS:2013jsu}. This study reveals that the ratios \TwoS/$\Upsilon$(1S) and \ThreeS/$\Upsilon$(1S) decrease with multiplicities. Collectively, these measurements converge in indicating that excited-state quarkonia are more responsive to final-state effects, with the degree of effect correlating with charged particle multiplicity.

QGP Signatures and Charmonia Behavior:
Beyond the realm of heavy-ion collisions, high-multiplicity $pp$ and $p$Pb collisions unveil signatures indicative of QGP's presence, such as strangeness enhancement~\cite{ALICE:2016fzo}, robust collective flow~\cite{CMS:2016fnw}, and long-range azimuthal correlations in two-particle correlations~\cite{CMS:2010ifv, ALICE:2012eyl, ATLAS:2012cix, LHCb:2015coe}. The potential manifestation of QGP could indeed impact the ratio of excited-to-ground state quarkonia. Excited energy-level quarkonia are anticipated to dissociate earlier in QGP due to their weaker binding compared to ground-state quarkonia. Consequently, studies that probe multiplicity dependence serve as valuable tests for QGP-like behaviors even in smaller collision systems.

Methodology and Analysis Overview:
The analysis methodology follows a structured sequence. Beginning with 8.16 TeV $p$Pb collision data from 2016, samples of $\jpsi$ and $\psitwos$ are isolated by fitting the invariant mass spectrum of oppositely charged muons. These samples are then separated into prompt and non-prompt components by fitting the pseudo proper-time across multiple multiplicity bins. To ensure accuracy, the yields are corrected using efficiency factors from various sources. Subsequently, the production ratios are computed within each multiplicity bin to obtain the final results. Three variables, namely PVNTRACKS, nForwardTracks, and nBackwardTracks, serve as proxies for collision multiplicity. While PVNTRACKS represents the global multiplicity of $pp$ collisions, nForwardTracks captures the multiplicity within the same phase space as the two charmonia. The inclusion of nBackwardTracks helps mitigate potential correlations between the ratio and nForwardTracks, thereby enhancing the analysis's robustness.

By integrating these improvements, your introduction should be more coherent and engaging, offering a comprehensive foundation for understanding charmonia production in collisions and its connection to the Quark-Gluon Plasma.




User
I also want to add introduction of Cold Nuclear Matter effect, and its influence in quarkonium production
ChatGPT
Certainly, introducing the Cold Nuclear Matter (CNM) effects and their impact on quarkonium production is an important addition to your introduction. Here's how you can seamlessly integrate this information:

Introduction to Cold Nuclear Matter Effects:
In the study of quarkonium production within heavy-ion collisions, it is essential to consider not only the dynamic interactions within the Quark-Gluon Plasma (QGP) but also the influences originating from Cold Nuclear Matter (CNM) effects. CNM effects refer to the interactions that occur between the constituents of the colliding nuclei before the QGP is formed. These effects can significantly affect the initial conditions and subsequent quarkonium production, imposing both modifications and constraints on the observed signals.

Interplay Between CNM Effects and Quarkonium Production:
The characteristics of CNM effects can be diverse, including shadowing, gluon saturation, and nuclear modification of parton distribution functions (nPDFs), among others. These CNM effects arise from the presence of nuclear structure and the inherent interactions between quarks, gluons, and the nuclear medium. They manifest through modifications in the initial-state parton distributions, which in turn influence the production rates and kinematics of quarkonia.

Importance of Separating CNM Effects from QGP Signatures:
Distinguishing between CNM effects and the subsequent QGP-driven dynamics is crucial for a precise understanding of quarkonium production. CNM effects can lead to both suppression and enhancement patterns, akin to what is observed in QGP-induced modifications. As such, a comprehensive analysis requires disentangling these effects from each other to unveil the distinct contributions and isolate the genuine QGP signatures.

Multiphase Approach to Quarkonium Production:
Incorporating CNM effects into the broader picture of quarkonium production necessitates a multiphase approach that encapsulates the full spectrum of nuclear interactions. This approach involves accounting for both initial-state and final-state effects, encompassing CNM effects, QGP-driven modifications, and the subsequent interactions between the produced quarkonia and the evolving medium.

Comprehensive Characterization of the Medium:
By meticulously considering the interplay between CNM effects and QGP dynamics, researchers gain a more comprehensive characterization of the medium formed in heavy-ion collisions. This holistic approach facilitates the disentanglement of distinct effects, enabling a more accurate interpretation of experimental observations and providing valuable insights into the properties of both CNM and QGP.

Incorporating these points should enhance your introduction by providing a well-rounded perspective on the interplay between Cold Nuclear Matter effects and the production of quarkonia in heavy-ion collisions, further enriching the context for your readers.
