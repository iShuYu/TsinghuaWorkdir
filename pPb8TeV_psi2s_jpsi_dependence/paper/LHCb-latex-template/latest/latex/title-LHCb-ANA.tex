% ===============================================================================
% Purpose: LHCb-ANA Note title page template
% Author: 
% Created on: 2010-10-05
% ===============================================================================

%%%%%%%%%%%%%%%%%%%%%%%%%
%%%%%  TITLE PAGE  %%%%%%
%%%%%%%%%%%%%%%%%%%%%%%%%
\begin{titlepage}

% Header ---------------------------------------------------
\vspace*{-1.5cm}

\noindent
\begin{tabular*}{\linewidth}{lc@{\extracolsep{\fill}}r@{\extracolsep{0pt}}}
\ifthenelse{\boolean{pdflatex}}% Logo format choice
{\vspace*{-1.2cm}\mbox{\!\!\!\includegraphics[width=.14\textwidth]{figs/lhcb-logo.pdf}} & &}%
{\vspace*{-1.2cm}\mbox{\!\!\!\includegraphics[width=.12\textwidth]{figs/lhcb-logo.eps}} & &}
 \\
 & & LHCb-ANA-XXX-XXX \\  % ID 
 & & \today \\ % Date - Can also hardwire e.g.: 23 March 2010
 & & \\
\hline
\end{tabular*}

\vspace*{4.0cm}

% Title --------------------------------------------------
{\normalfont\bfseries\boldmath\huge
\begin{center}
% DO NOT EDIT HERE. Instead edit macro in main.tex to keep metadata correct
  \papertitle
\end{center}
}

\vspace*{2.0cm}

% Authors -------------------------------------------------
\begin{center}
% If changing to list here, make pdfauthors in main.tex a comma
% separated list with the same names. Otherwise metadata in file will be wrong.
\paperauthors.
\bigskip\\
{\normalfont\itshape\footnotesize
  Tsinghua University\\
}
\end{center}

\vspace{\fill}

% Abstract -----------------------------------------------
\begin{abstract}
  \noindent
	Using a data sample with an integrated luminosity of 13.6 nb$^{-1}$ for $p$Pb collisions and 20.8 nb$^{-1}$ for Pb$p$ collisions collected by the LHCb detector in the LHC operations in 2016, ratio of production cross section of $\psi(2S)$ over $J/\psi$ as a function of multiplicity was measured at a centre-of-mass energy $\sqrt{s_{NN}}=8.16$ TeV. 
	A multiplicity-dependent modification of the ratio has been observed for prompt mesons in $p$Pb collisions when there is an overlap between the rapidity ranges where the multiplicity and the charmonia production are measured. No evident modification of same significance for that of non-prompt component is observed. In Pb$p$ collisions, there is a overall reduction in the ratio compared to that in $p$Pb collisions, but no obvious decreasing trend on the ratio is observed for both prompt and non-prompt mesons.
	The ratio as a function of multiplicity is compared to co-mover model. The ratio measured in $p$Pb and Pb$p$ collisions are compared to other measurements in different collision systems, which shows a good agreement.
	
\end{abstract}

\vspace*{2.0cm}
\vspace{\fill}

\end{titlepage}


\pagestyle{empty}  % no page number for the title 

%%%%%%%%%%%%%%%%%%%%%%%%%%%%%%%%
%%%%%  EOD OF TITLE PAGE  %%%%%%
%%%%%%%%%%%%%%%%%%%%%%%%%%%%%%%%

%  empty page follows the title page ----
\newpage
\setcounter{page}{2}
\mbox{~}


