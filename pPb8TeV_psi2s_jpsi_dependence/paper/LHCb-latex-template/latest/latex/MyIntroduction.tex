\section{Introduction}
\label{sec:Introduction}
Quark-Gluon Plasma (QGP)~\cite{Satz_2011} emerges as a distinct state of matter characterized by extreme temperatures and densities. Within QGP, quarks and gluons undergo deconfinement, gaining asymptotic freedom. This exotic phase is believed to have been present shortly after the Big Bang, and it's now under scrutiny through heavy-ion collisions. These collisions, facilitated by ultra-relativistic energies, create conditions akin to those in the early universe, offering a potential window into QGP. Within this context, the production of quarkonia, specifically charmonia, gains prominence as a valuable probe for QGP due to its rapid production timeline relative to the timescales of QGP formation.

In comprehending the intricate nexus between charmonia production within collisions and its correlation to Quark-Gluon Plasma, a comprehensive consideration must encompass dynamic QGP interactions alongside the influences stemming from Cold Nuclear Matter (CNM) effects, as CNM effects can mirror patterns of suppression or enhancement reminiscent of QGP-induced alterations, an all-encompassing analysis becomes imperative, enabling the separation of their distinctive contributions and the elucidation of their unique signatures. CNM effects encompass interactions that transpire prior to QGP formation, thereby exerting a profound impact on initial conditions and ensuing quarkonium production. These effects encapsulate a spectrum of phenomena, including shadowing, gluon saturation, and nuclear modification of parton distribution functions (nPDFs), all arising from nuclear structure and inherent quark-gluon-medium interactions. Hence, measurements in $p$Pb collisions serves as a benchmark which enables us to study CNM effects with the absence of QGP.

The intricacies of quarkonia production are entwined with the evolution of heavy-ion collisions. Initial-state conditions play a pivotal role, encompassing nuclear parton distribution functions (nPDFs) for lead nuclei and parton distribution functions for protons~\cite{AtashbarTehrani:2017mzi}, as well as energy loss due to multi-particle scattering~\cite{Arleo:2014oha}. These effects exert substantial influence over quarkonium production, potentially leading to enhancements or suppressions in their yields. When examining the ratio of excited-to-ground state quarkonia production, certain initial-state effects are anticipated to cancel out. However, empirical evidence points to preferential suppression of excited states of charmonia (\psitwos) and bottomonia (\TwoS, \ThreeS) compared to their ground states (\jpsi and $\Upsilon$(1S))\cite{LHCb:2018psc,ALICE:2020vjy, LHCb:2016vqr}. These observations underscore the involvement of additional final-state effects, such as interactions with co-moving particles\cite{Ferreiro:2012rq}.

Recent measurements of \jpsi and \psitwos production in $p$Pb collisions at $\sqrt{s_{NN}}=8.16$ TeV~\cite{ALICE:2020tsj} spotlight the heightened sensitivity of \psitwos to final-state interactions. Particularly noteworthy is the significantly lower nuclear modification of \psitwos compared to \jpsi, especially in intermediate multiplicity classes. The CMS collaboration's investigation into $\Upsilon$(nS) production in $p$Pb collisions at $\sqrt{s_{NN}}=5.02$ TeV and $pp$ collisions at $\sqrt{s}=2.76$ TeV~\cite{CMS:2013jsu} further uncovers multiplicity-dependent ratios (\TwoS/$\Upsilon$(1S) and \ThreeS/$\Upsilon$(1S)). Taken collectively, these measurements emphasize the heightened responsiveness of excited-state quarkonia to final-state effects, with the extent of their influence intricately tied to the multiplicity of charged particles.


As the analysis unfolds, the process involves the selection of samples from 8.16 TeV $p$Pb collision data collected in 2016, where $\jpsi$ and $\psitwos$ are identified through fitting the invariant mass spectrum of oppositely charged muons. Furthermore, the distinction between prompt and non-prompt components is established by fitting the pseudo proper-time within various multiplicity bins. Ensuring precision necessitates correcting yields using efficiency factors from diverse sources. Subsequently, the production ratios are calculated across distinct multiplicity bins to attain conclusive results. Notably, three variables—$N^{\rm PV}_{\rm tracks}$, $N^{\rm PV}_{\rm fwd}$, and nBackwardTracks—operate as proxies for collision multiplicity. In particular, $N^{\rm PV}_{\rm tracks}$ captures the global multiplicity of $p$Pb collisions, whereas nBackwardTracks corresponds to the number of tracks in the backward direction within the rapidity range $-5.2 < \eta < -1.5$, and $N^{\rm PV}_{\rm fwd}$ pertains to the number of tracks in the forward direction with the rapidity range $1.5 < \eta < 5.2$. Since correlations between the ratio and $N^{\rm PV}_{\rm fwd}$ may arise due to co-moving effects, the inclusion of nBackwardTracks mitigates this impact, thereby enhancing the analysis's sensitivity to the potential existence of QGP.

