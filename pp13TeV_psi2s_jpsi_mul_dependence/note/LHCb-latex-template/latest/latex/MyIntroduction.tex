\section{Introduction}
\label{sec:Introduction}
In normal matter, quarks and gluons are confined within particles called hadrons, such as protons and neutrons. However, at extremely high temperatures and densities, such as those that existed shortly after the Big Bang or in high-energy particle collisions, the strong force that binds quarks and gluons together becomes weaker. It results in the creation of a new form of matter Quark Gluon Plasma (QGP) in which the partons are evolving quasi freely. The measurement of quarkonia suppression is considered a probe of QGP. In such a hot and dense state, the existence of a large density of free color charges will lead to a color screening of quark and anti-quark binding, hence, the dissociation of quarkonium~\cite{MATSUI1986416}.

But before we interpret any phenomena observed in hot, dense QCD matter, baseline measurements need to be performed in the small system such as $pp$ collisions in which no QGP is in principle produced. With the increasing charged-particle multiplicity, a transition from normal density to a relatively high-density environment serves as a probe for how quarkonium production suppression is affected by the charged particle multiplicity. 

Quarkonia suppression can be detected by measuring nuclear modification factors. This factor is computed by measuring quarkonia yields in heavy-ion collisions and compared to the yield measured in $pp$ collisions scaled by the number of nucleons. The PHENIX collaboration measured the nuclear modification factor of $\jpsi$ in dAu collisions at 200 GeV~\cite{PHENIX:2007tnc} and observed a significative suppression while no QGP production is expected. More recent results by LHCb also confirm this suppression ~\cite{LHCb:2017ygo} in the nuclear modification factor measured using $p$Pb collisions. However, this suppression can be explained by different effects such as the nuclear Parton Distribution functions (nPDFs) of the colliding heavy-ion~\cite{AtashbarTehrani:2017mzi}, quarkonia energy loss~\cite{Arleo:2014oha} or interaction with co-moving particles~\cite{Ferreiro:2012rq}. The latter effect would affect quarkonia from two excited states differently, as they have different binding energy, and would have a stronger influence when the multiplicity of the collisions is high. The effect governing heavy quark production is expected to be similar for charmonium with the same content.

As we mentioned above, quarkonia suppressions in $pp$ collisions have been traditionally served as a baseline measurement when searching QGP. While more and more signatures of QGP has been observed in small systems such as high-multiplicity $pp$ collisions. For examples, ALICE collaboration has reported a measurement of strangeness enhancement in high-multiplicity $pp$ collisions at $\sqrt{s}=7$ TeV~\cite{ALICE:2016fzo} and CMS collaboration found a high degree of collectivity flow in high-multiplicity $pp$ collisions at $\sqrt{s}=13$ TeV~\cite{CMS:2016fnw}. The possible existence of QGP in high-multiplicity $pp$ collisions could also have an effect on the suppressions of charmonia since color Debye Screening effect will prevent quark from forming a quarkonium with an anti-quark. Quarkonia with same content but different energy levels may dissociate at different temperatures. Those with higher energy levels have a weaker bound will dissociate first compared to the lower-levels in QGP, which will cause a similar effect on the ratio of $\sigma_{\psitwos}/\sigma_{\jpsi}$. Hence, a further calculation from co-mover model prediction is needed to verfy the sources of different suppressions, if exist.
    
The analysis proceeds as follows: Samples of $\jpsi$ and $\psitwos$ are selected from 13 TeV
$pp$ collisions data collected in 2016, by fitting the invariant mass spectrum of oppositely
charged muons. Prompt and non-prompt components are separated by fitting the pseudo
proper-time in multiple bins of multiplicity. Then the yields are corrected by the efficiencies
from different sources. Then the ratio of production is calculated in each multiplicity bin to
come to a final result. Three variables, PVNTRACKS, nForwardTracks and nBackwardTracks,
are used as proxy for the multiplicity of the $pp$ collisions. PVNTRACKS is the number of tracks
used to reconstruct the primary vertex, and nBackTracks is the number of tracks in the
backward directions. PVNTRACKS is the global multiplicity of the pp collisions while
nForwardTracks is the multiplicity measured in the same phase space as the two charmonia.
Some correlations between the ratio and nForwardTracks can appear and due to that, the
use of nBackTracks, allows to reduce this effect.
