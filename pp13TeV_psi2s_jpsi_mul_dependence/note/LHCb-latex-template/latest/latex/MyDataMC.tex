\section{Data and Monte Carlo samples}
\label{Data and Monte Carlo samples}
\subsection{Data}
The study here uses $pp$ collision data collected by the LHCb detector at a center-of-mass energy of 
13 TeV in 2016 with a corresponding luminosity of 658$\pm$13 pb$^{-1}$. To enlarge the sample size, two different 
TCKs 0x1138160F and 0x11381612 of both MagUp and MagDown were used. In this analysis, only muon triggers are 
used, and both TCKs have identical criteria for muon selection. The bookkeeping paths for the mdst file for both \jpsi and \psitwos are as follows:
\begin{itemize}
    \item MagUp /LHCb/Collision16/Beam6500GeV-VeloClosed-MagUp/Real Data/Turbo03a/94000000/LEPTONS.MDST
    \item MagDown /LHCb/Collision16/Beam6500GeV-VeloClosed-MagDown/Real Data/Turbo03a/94000000/LEPTONS.MDST
\end{itemize}
The NoBias data is used for normalization of multiplicity. The bookkeeping path for the mdst file is:
\begin{itemize}
    \item /LHCb/Collision16/Beam6500GeV-VeloClosed-MagDown/Real Data/Reco16/96000000/FULL.DST
\end{itemize}


\subsection{Monte Carlo}
To study the efficiency, full simulation Monte Carlo samples with about 20 M candidates for \jpsi and 10 M for \psitwos. The bookKeeping path for them are:
\begin{itemize}
    \item \jpsi
    \begin{itemize}
        \item MagUp /MC/2016/24142001/Beam6500GeV-2016-MagUp-Nu1.6-25ns-Pythia8/Sim09b/Trig0x6138160F/Reco16/Turbo03/ \\
        Stripping26NoPrescalingFlagged/ALLSTREAMS.DST
        \item MagDown /MC/2016/24142001/Beam6500GeV-2016-MagDown-Nu1.6-25ns-Pythia8/Sim09b/Trig0x6138160F/Reco16/Turbo03/ \\
        Stripping26NoPrescalingFlagged/ALLSTREAMS.DST
    \end{itemize}
    \item \psitwos
    \begin{itemize}
        \item MagUp: /MC/2016/28142001/Beam6500GeV-2016-MagDown-Nu1.6-25ns-Pythia8/Sim09i/Trig0x6139160F/Reco16/Turbo03a/ \\
        Stripping28r2NoPrescalingFlagged/ALLSTREAMS.DST
        \item MagDown: /MC/2016/28142001/Beam6500GeV-2016-MagUp-Nu1.6-25ns-Pythia8/Sim09i/Trig0x6139160F/Reco16/Turbo03/ \\
        Stripping28r2NoPrescalingFlagged/ALLSTREAMS.DST
    \end{itemize}
\end{itemize}
In the simulation, $pp$ collisions are generated using Pythia~\cite{Ball:2006wn} with a specific LHCb configuration~\cite{LHCb:2011dpk}. Decays of hadronic 
particles are described by EvtGen~\cite{Lange:2001uf}, in which final state radiation is generated using Photos~\cite{Golonka:2005pn}. The 
interaction of the generated particles with the detector and its response are implemented using the Geant4 
toolkit~\cite{GEANT4:2002zbu} as described in Ref.~\cite{Clemencic:2011zza}. The prompt charmonium production is simulated in Pythia with 
contributions from both the leading order color-singlet and color-octet mechanisms~\cite{LHCb:2011dpk,Bargiotti:2007zz}, and the charmonium 
is generated without polarization.
To study the geometrical acceptance, two samples of 100 k candidates generator level Monte Carlo are produced for both \jpsi and \psitwos respectively. Since the 
acceptance is only a function of kinematic variables, hence, it is universal for all multiplicity regions. Under the binning scheme in Sec~\ref{Ratio of Cross-Section Determination}, the sample size is fairly enough.

